\section{Lightning and Thunderstorms}

%% Figure differentiating different stroke types
%% Write

\section{The Ionosphere}

%% Borrow / adapt electron density type figure
%% Write

\section{Very Low Frequency Wave Propagation}

%% Propagation figure
%% Get from Azimuth

\section{Global Electric Circuit}

%% Diagram of GEC
%% Get from GEC paper

\section{Lightning Detection Systems}

%% Get paragraph about lightning detection uses overall from ENTLN-LIS

\subsection{World Wide Lightning Location Network}

%% Add in global stroke distribution figure

%% Update with more recent papers
The World Wide Lightning Location Network (WWLLN, see http://wwlln.net) determines the location for nearly all lightning producing storms around the globe in real time (c.f. \citet{Jacobson2006c}).
WWLLN has been generating global lightning locations starting in 2004 \citep{Rodger2006, Rodger2009}.
Since then the network has grown from 18 stations to over 70 as of September 2013.
Knowledge of individual stroke locations, with high temporal accuracy, and within a fraction of a wavelength is beneficial for both scientific and technical uses.
WWLLN lightning location data have recently been used for advances in space science \citep{Lay2007, Kumar2009, Collier2009, Holzworth2011, Jacobson2011}, meteorology \citep{Price2009,Thomas2010d}, detailed lightning physics \citep{Connaughton2010}, and volcanic eruption monitoring \citep{Doughton2010}.

The network uses a time of group arrival (TOGA) technique, originally developed by \citet{Dowden2002d}, to locate strokes by analyzing the sferic waveforms at each station using the Stroke\_B algorithm as discussed by \citet{Rodger2006,Rodger2009}.
WWLLN locates strokes by analyzing the TOGA of the sferic wave packet in the 6 -- 18~kHz band \citep{Dowden2000}.
The TOGA of the VLF wave packet, is used rather than ``trigger time'' to produce more uniform arrival times across the network.
A recent upgrade to the network allows for the measurement of the radiated VLF energy of located strokes within the 8 -- 18~kHz VLF band.
The stroke energy is calculated from the time-integrated root mean square (RMS) electric field of the VLF sferic at each WWLLN station, using the Long Wave Propagation Capability code (LWPC) \citep{Ferguson1998} to estimate the sferic attenuation and calculate the source radiated energy \citep{Hutchins2012}.

As stations are added the accuracy and detection efficiency of the network improves.
As of 2010 the network locates most strokes to within 10 kilometers and $<$10~$\mu$s with an estimated detection efficiency of about 11\% for all strokes and $>$30\% for more powerful strokes \citep{Abarca2010,Rodger2009}.
The network improves in accuracy and detection efficiency with increased stations; for example an increase in the number of WWLLN stations from 11 in 2003 to 30 in 2007 led to a $\sim$165\% increase in the number of lightning strokes located \citep{Rodger2009}.
However the WWLLN network does not observe lightning with the same detection efficiency everywhere.
This is due to variable WWLLN station coverage and the strong affect on VLF radio propagation from orography and ionospheric conditions along the great circle path of a wave.

\subsection{Earth Networks Total Lightning Network}

%% Add ENTLN stroke density figure

The Earth Networks Total Lightning Network (ENTLN) is a ground based network that began in 2009 as the Weatherbug Total Lightning Network (WTLN).
It has two operational regimes: short range using broadband sferic waveforms (5~kHz -- 12~MHz) and long range with only VLF waveforms (1~Hz -- 256~kHz) \citep{Heckman2010}.
Correlations of the stroke waveform and amplitude from multiple stations determines the time, location, altitude, peak current, polarity, and type of the located stroke \citep{Liu2011a}.
The network utilizes a time of arrival method to determine the location of each stroke, where a minimum of 8 stations is required to produce a valid solution.
To compress the broadband waveforms each station removes the necessary amount of low-amplitude signal to reach the requisite packet size.
In the continental United States (CONUS) the network has approximately 530 operational stations in September 2013.
A comparison to the Oklahoma Lightning Mapping Array (OKLMA) in 2010 demonstrated the ability to discriminate between CG and IC strokes \citep{Beasley2010}.

\subsection{TRMM/LIS}

%% Add LIS global flash count figure.

The Lightning Imaging Sensor (LIS, 1997-present) is a satellite-based lightning detector flown onboard the Tropical Rainfall Measurement Mission (TRMM) satellite orbiting at a 35$^\circ$ inclination and 402~km altitude \citep{Christian1999}.
In low earth orbit it observes the total lightning activity from individual thunderstorms for 90 sec in each $0.5^\circ \times 0.5^\circ$ viewtime granule.
LIS is useful as it is a lightning detection system with no overlapping detections methods with ground based networks, and the sensor performance has not changed over time.
The LIS data are available at several processed levels, throughout this work the flash level data is used.

\section{Long Wave Propagation Capability Code}

%% Combine all LWPC descriptions here
%% Lookup figure
