The process for bootstrap calibrating the WWLLN stations and how the energy is calculated is described in \citet{Hutchins2012}.
This section walks through the code used in the processing along with the decisions behind various checks and adjustments made in the processing.

\section{Code Summary}

There are XXX main sections to the bootstrap processing:

\begin{itemize}
	\item{Process R-files into AP-files}
	\item{Calculate LWPC attenuation coefficients for each stroke-station pair}
	\item{Bootstrap calibrate the network}
	\item{Calculate stroke energy using LWPC and calibration}
	\item{Iterative re-calibrate the network}
	\item{Final energy calculation}
	\item{Run Relative Detection Efficiency Model (Chaper~\ref{ch:de})}
	\item{Move files to storage locations}
\end{itemize}

This processing requires several files, locations given relative to mlhutch@flash5.ess.washington.edu.

\begin{itemize}
	\item{\textbf{$\sim$/Process/relocate-B31Jan2013.x86-64} -- James Brundell's relocate program used to process R-files into AP-files, this version compiled for Linux 64-bit systems}
	\item{\textbf{stations.dat} -- list of current WWLLN stations, copied over from flash4 at the start every processing run.}
	\item{\textbf{lwpcv21} -- directory containing a working compiled version of the LWPC code.
		Used in generating new lookup tables as stations are added, a parallelized matlab implemtation is discussed below.}
	\item{\textbf{$\sim$/matlab/Bootstrap/} -- directory containing the necessary matlab files, available as a git directory (see Appendix~\ref{app:code}).}
	\item{\textbf{$\sim$/matlab/functions/} -- directory of necessary matlab functions used by the Bootstrap code files, also available as a git directory (see Appendix~\ref{app:code}).}
\end{itemize}

\section{LWPC}

The Long Wave Propagation Capability code is a codebase developed by \citet{Ferguson1998} that is used to calculated the electric field at a given location for a VLF transmitter at another location.
For the WWLLN energy processing it is used to estimate the attenuation between a stroke (treated as a transmitter) and a station.
The original LWPC code has been altered in two ways: first the Windows-compiler specific code has been replaced with GCC compilable code, second all warning have been removed to produce output of a constant shape (for reading into MATLAB).
All edits in the source code have been marked by my initials of MH.

\subsection{Compiling}

To recompile LWPC two bash scripts need to be run.
The first is \textbf{BuildData.cmd}, BuildData recompiles the data files that contain the surface parameters such as ground conductivity and coastlines.
The second step is to run \textbf{buildlwpc.cmd}, this should compile the program and result in an executable called LWPC.
To test a successful compile run the script \textbf{run\_bench.cmd}, if it runs successfully it is compiled, if not consult the \textbf{Readme - Unix.txt} file that has some common troubleshooting steps.

The last step in setting up LWPC is to set the \textbf{lwpcDAT.loc} file to point towards the \textbf{data} folder.

\subsection{Running}

LWPC is run from the command line with the structure:

\begin{verbatim}
./LWPC test1.inp
\end{verbatim}

Where \textbf{test1.inp} is formatted as per the structure in \textbf{User\_manual.pdf}.
The matlab implemtations discussed below automatically generate formatted input files.

The result of running LWPC is an output file the specifies the electric field at various distance along the path from the transmitter to receiver.
It is also capable of outputting plots, azimuthal dependence, and many other features not utilized in this research.

\subsection{Matlab Function}

A matlab implementation of LWPC is available as a git repository (see Appendix~\ref{app:code}).
This implantation allows for LWPC to called in matlab with:

\begin{verbatim}
LWPCpar(freq,lat,long,time,stat_lat,stat_long,model)
\end{verbatim}

Where freq is the transmitter frequency, lat/long is the transmitter location, time is the date and time, stat\_lat/stat\_long are the receiver locations, and model is the ionospheric model used.
If model is set to ``time'' then day and night is considered in the calculation, if it is set to ``day'' or ``night'' an all day or all night ionosphere will be used.

LWPCpar as the advantage of being runnable within matlab parallelized loops (parfor).
While LWPC itself cannot be run in parallel, this method simply copies the LWPC directory to allow each matlab instance it's own copy.

\section{Lookup Tables}

\subsection{Description}

\subsection{Generation}

And automation

\subsection{Code}

\section{Bootstrap and Calibration Process}

\section{Matlab Code}

\subsection{Rerunning Lost Days}

\subsection{Automation}

\subsection{Time Estimates}

\section{Relative Detection Efficiency Code}

\subsection{Point to Paper}

\subsection{Code Summary}

\subsection{Code Description and Location}

\subsection{Output Interpretation}

\subsection{Matlab Code}