This appendix discusses the distribution of WWLLN data files on the Flash series of computers along with the storage locations of routine processing code.
The code locations are given in terms of both their active locations and as the source control repositories, managed using git (http://git-scm.com).

\section{File Storage on Flash Machines}

There are several types of WWLLN data files, summarized in Table~\ref{code:fileType}.
The example filenames given in the table are for 31 December 2013, 5:10 UTC.
Most of the file types, with exception of WB-files, are located on flash5 on one of two hard drives.
Data for the current year is stored on /wd2/ while previous years are stored on /wd3/.
Each file type has it's own directory, denoted Afiles, AEfiles, etc., with flat file structures (except for R-files).
R-files, due to the large number generated, are broken into yearly and monthly folders of the form: /wd2/R[YYYY]/R[YYYY][MM]
The R-file repository on flash5 differs from that of flashdat since the 2005 and 2006 data has been cleaned and organized into a consistent file hierarchy.
Some, but not all, of the WWLLN data file directories have been copied to the other flashes on a need-basis and so the other flashes should not be considered a backup of the flash5 dataset.

\begin{table}[h!]
\begin{center}
\caption{WWLLN Data Types}
\begin{tabular}{|p{.75in}|p{2in}|p{3.5in}|}

\hline
{\bf File Type} & {\bf Naming} &	{\bf Description} \\

\hline
\rule{0pt}{3ex}
R-files	& R20121231051000	&	Raw packets sent by stations\\ 

\hline
\rule{0pt}{3ex}
A-files	& A20121231.loc	&	Lightning locations generated in real time\\ 

\hline
\rule{0pt}{3ex}
AP-files	& AP20121231.loc	&	Location data with station and E-field data of each stroke solution, generated daily with relocate-B\\ 

\hline
\rule{0pt}{3ex}
AE-files	& AE20121231.loc	&	Location and stroke energy data, generated daily by Bootstrap (Chapter~\ref{ch:energy})\\ 

\hline
\rule{0pt}{3ex}
DE-files	& DE20121231.mat	&	Relative detection efficiency maps, as Matlab data (Chapter~\ref{ch:de})\\ 

\hline
\rule{0pt}{3ex}
S-files	& S201212310510.Seattle.loc	&	Waveform data for strokes located near a given station.\\ 

\hline
\rule{0pt}{3ex}
WB-files	& WB201212310510.dat	&	Continuos wideband field data output by toga -r\\ 

\hline
\rule{0pt}{3ex}
T-files	& T20130130.dat	&	Station packet count record, listed date is last date updated-r\\ 

\hline
\end{tabular}
\end{center}
\label{code:fileType}
\end{table}
 
An example line from each of the R, A, AP, and AE files are given below with descriptors for each file.
R-files are the simplest with each line containing three columns:

\begin{verbatim}
10      3000.015563     2668
\end{verbatim}

The first number is the station ID, the second is the seconds from the start of the hour, and the last is the root integrated square electric field at the station, in uncalibrated sound card units.
For A-files each line if of the form:

\begin{verbatim}
2013/01/31,16:18:54.103460,  9.0621,-117.8276, 19.9,  9
\end{verbatim}

The first six numbers are the date (YYYY/MM/DD) and UTC time of the stroke (hh,mm,ss).
The next two are the stroke location in degrees north latitude and east longitude.
The final two are the timing uncertainty in microseconds and the number of WWLLN stations that participated in the stroke location.
AP-files are similar to A-files with additional information after the stroke location:

\begin{verbatim}
2013/02/12,00:00:07.142335,-10.4622,  20.0863, 13.4,  5,17,600,19,2084,26,453,53,225,
	64,2263
\end{verbatim}

Starting at 17,600 are pairs of numbers that give the station ID (17) and the field strength (600) in sound card units, the same value as in the R-files. Finally AE-files are similar to A-files but with three additional numbers:

\begin{verbatim}
2013/2/12,00:00:07.142335, -10.4622,  020.0863, 13.4, 5, 448.86, 114.87, 4
\end{verbatim}

The last three numbers correspond to the radiated VLF stroke energy in joules (448.86), the median absolute deviation of the energy value in joules (114.87), and the subset of stations that participated in the energy value solution (4).

T-files are records of how many packets a given station sent to flash4 during the history of WWLLN.
Unlike the other files it is not automatically generated, rather it is written whenever it is updated through the TfileUpdater.m script in the functions.git repository.
Each line contains the date in days since 01/01/0000 and the counts for each station, where column 2 corresponds to station 0.

\section{Code Repositories}

All of my code should be stored on flash5 at /home/mlhutch/matlab for Matlab scripts.
For example the code to run the energy processing is located at /home/mlhutch/matlab/Bootstrap/bootstrap\_automation.m.
The exact details of where the energy processing code is located, the necessary files, and similar code for the relative detection efficiency processing is covered in Appendix~\ref{app:energy}.
Aside from being stored in plaintext scripts, all of the code is also stored in a Git repository located both on flash5 (/home/mlhutch/Git) and hosted online on bitbucket.org.

The Git repositories have the advantage of multiple back ups, version control, easily portable, and easy to use.
A list of the WWLLN processing code and corresponding git repository is given in Table~\ref{code:repo}.
A basic primer on how to use git, including retrieval of the repositories, is given in Section~\ref{code:primer}.

\begin{table}[h!]
\caption{Git Repositories (flash5:/home/mlhutch/Git/)}
\begin{center}
\begin{tabular}{|p{1.5in}|p{1.25in}|p{3in}|}

\hline
{\bf Code} &	{\bf Repository} &	{\bf Notes}\\

\hline
\rule{0pt}{3ex}
Energy Processing	&bootstrap.git	&	Requires process.git and functions.git \\ 

\hline
\rule{0pt}{3ex}
Detection Efficiency	&de.git	&	Requires functions.git\\ 

\hline
\rule{0pt}{3ex}
LWPC	&lwpc.git	&	LWPC and Matlab implementation\\ 

\hline
\rule{0pt}{3ex}
AP Processing	&process.git	&	Relocate files (required for energy processing)\\ 

\hline
\rule{0pt}{3ex}
MATLAB Functions	&functions.git	&	Various Matlab functions required by other scripts.\\ 

\hline
\rule{0pt}{3ex}
SU Eagle Files	&eagle.git	&	Eagle files for WWLLN SU and Pre-amps, includes Erin Lay's files.\\ 

\hline
\rule{0pt}{3ex}
Gumstix	&gumstix.git	&	Files for building and configuring Gumstix microSD cards.\\ 

\hline
\rule{0pt}{3ex}
Gumstix Bitbake	&openembedded.git	&	Bitbake distribution for adjusting Gumstix kernel\\ 

\hline
\end{tabular}
\end{center}
\label{code:repo}
\end{table}

\section{Git Primer}
\label{code:primer}

Git (http://git-scm.com) is an open source distributed version control system available on all operating systems with very little system resources required.
Most often git is accessed through the command line but there are many graphical interfaces available.
I chose git for my research and code since it allows easy branching, constant backups (locally and on remote servers), and helps to organize code and projects.
A brief tutorial on git is available online: http://git-scm.com/book/en/Getting-Started and http://try.github.com/.

Once git is installed on a host machine (http://git-scm.com, apt-get install git, or yum install git-core) repositories can be downloaded with the command:

\begin{verbatim}
git clone [user]@[server]:[repository] .
\end{verbatim}

This will copy (clone) the repository to the current directory as a folder of the same name as the repository. 
For example to copy over the energy processing repository listed in Table~\ref{code:repo} to the matlab directory:

\begin{verbatim}
cd ~/matlab
clone sferix@flash5.ess.washington.edu:/home/mlhutch/Git/bootstrap.git .
\end{verbatim}

Where the user can be anyone with permission to access that directory.
More information on how to use git for source control, instead of downloading the repositories, can be found in the online resources.
I am happy to have user collaborate with the flash5 git repositories for updates, changes, or additions to the listed repositories.

\section{Other Useful Matlab Code}
