\section{Creating Gumstix microSD Card}

\begin{verbatim}
# Make and populate Gumstix microSD card for WWLLN

# Copy necessary files from ~/bbImages to ~/Work/ and rename
#cp ~/bbImages/MLO-overo ~/Work/bitbake/MLO
# Use updated MLO instead
#cp ~/Work/Angstrom/mlo-updated ~/Work/bitbake/MLO
#cp ~/bbImages/uImage-overo.bin ~/Work/bitbake/uImage
#cp ~/bbImages/u-boot-overo.bin ~/Work/bitbake/u-boot.bin
#cp ~/Work/Angstrom/uImage ~/Work/bitbake/uImage
#cp ~/Work/Angstrom/u-boot.bin ~/Work/bitbake/u-boot.bin
#cp ~/bbImages/omap3-console-image-overo.tar.bz2 ~/Work/bitbake/rootfs.tar.bz2

# Card size in bytes /255 /63 /512 to rounded down to get cylinders
IMG=sakoman-gnome-image.tar.bz2
UIMAGE=uImage-3.0-r102-omap3-multi.bin
SD=/dev/sdb
DIR=~/gumstix/baseKernel/
MOD=modules-3.0-r102-omap3-multi.tgz

echo Loading ${DIR}${IMG} onto $SD

echo 'Creating Parition Table'
# Partition the card
dd if=/dev/zero of=$SD bs=1024 count=1024

# Count Cylinders
SIZE=`fdisk -l $SD | grep Disk | awk '{print $5}'` 
echo DISK SIZE - $SIZE bytes
CYL=`echo $SIZE/255/63/512 | bc`
echo CYLINDERS - $CYL

# Make partition table
{
echo 128,130944,0x0C,*
echo 131072,,,-
} | sfdisk --force -D -uS -H 255 -S 63 -C $CYL $SD

echo 'Waiting for partition to settle'
sleep 1
echo '.'
sleep 3
echo '..'
sleep 3
echo '...'
sleep 3


echo 'Formatting and Mounting'
#Format and Mount
mkfs.vfat -F 32 ${SD}1 -n boot
mke2fs -j -L rootfs ${SD}2
sleep 2
rm -r /media/boot
rm -r /media/rootfs
sleep 1
mkdir /media/boot
mkdir /media/rootfs
sleep 1
mount -t vfat ${SD}1 /media/boot
mount -t ext3 ${SD}2 /media/rootfs
sleep 1

# Copy over boot files and expand OS
echo 'Copying boot files and rootfs'
cp ${DIR}MLO /media/boot/MLO
cp ${DIR}u-boot.bin /media/boot/u-boot.bin
cp $DIR$UIMAGE /media/boot/uImage
tar xaf $DIR$IMG -C /media/rootfs
tar xzvf $DIR$MOD
rm -rf /media/rootfs/lib/modules
rm -rf /media/rootfs/lib/firmware
cp -r lib/modules lib/firmware /media/rootfs/lib/
rm -rf lib

sync

# Copy over vim syntax folder
tar -zxvf ${DIR}syntax
cp -r ${DIR}syntax /usr/share/vim/vim72/ 

# Adjust Network and opkg parameters
echo 'Adjusting parameters'
echo 'src/gz angstrom-base http://feeds.angstrom-distribution.org/feeds/core/ipk/eglibc/armv7a/base' > /media/rootfs/etc/opkg/angstrom-base.conf
NET=/media/rootfs/etc/network/interfaces

cp $NET ${NET}.defaults
cp firstRunFiles/interfaces $NET

# Copy first run script and files
echo 'Copying firstRun script and files'
# cp firstRun.sh /media/rootfs/home/root/
cp -r firstRunFiles /media/rootfs/home/root/

# Unmount
echo 'Unmounting microSD card'
umount /media/boot
umount /media/rootfs

rmdir /media/boot
rmdir /media/rootfs

echo 'Load into Gumstix and run /home/root/firstRun.sh'
\end{verbatim}


\section{Setting Gumstix OS Parameters}

\subsection{Installing Packages}

\begin{verbatim}

# Install necessary system packages

# Comment dhclient out if setting up via SSH
echo 'Starting DHCP'
dhclient

echo 'Installing packages...'
opkg update
opkg install gcc
opkg install libstdc++6
opkg install python-pyserial
opkg install gd
opkg install ntp
opkg install ntp-bin
opkg install gps-utils
opkg install vim
opkg install iptables
opkg install openvpn
opkg install vpnc
opkg remove apache2
opkg install apache2
opkg install openssh-keygen
opkg install openssh-ssh
#opkg install openssl
#opkg install openssh
# opkg install gpsd

# opkg remove dropbear --force-removal
# opkg install openssh

# Update and set clock
echo 'Setting clock'
/usr/bin/ntpdate -s -u bigben.cac.washington.edu

# echo 'System update'
# opkg upgrade

echo 'Run setup.sh'

\end{verbatim}

\subsection{Configuring Settings}

\begin{verbatim}
# Change Root password
echo 'Changing Root Password:'
passwd

# Setup Sferix user
echo 'Creating sferix account'
adduser sferix
usermod -s /bin/bash -G sferix sferix
usermod -a -G audio sferix
usermod -a -G dialout sferix
touch sferix_sudo
echo 'sferix ALL=(ALL) ALL' >> sferix_sudo
chmod 0440 sferix_sudo
cp sferix_sudo /etc/sudoers.d/
rm sferix_sudo

# Set user path
echo PATH=${PATH}:/home/sferix/bin > /home/sferix/.profile
echo export PATH >> /home/sferix/.profile

# Copy over configuration files
echo 'Copying Configuration Files'
DIR=''
cp ${DIR}sshd_config /etc/ssh/
#cp ${DIR}gpelogin /etc/sysconfig/
cat ${DIR}.bashrc >> /home/sferix/.profile
cp ${DIR}.bashrc ~/.profile
cp ${DIR}.vimrc ~/
cp ${DIR}.vimrc /home/sferix/
cp ${DIR}iptables /etc/iptables.rules
cp ${DIR}resolv.conf /etc/
cp ${DIR}ntp.conf /etc/
cp ${DIR}NetworkManager.conf /etc/NetworkManager/
rm /etc/rc*.d/*NetworkManager
cp ${DIR}dropbear /etc/init.d/
cp ${DIR}httpd.conf /etc/apache2/
cp ${DIR}gpsd /etc/default/
cp ${DIR}networkConfig.sh /home/sferix/
cp ${DIR}asound.state /etc/
cp ${DIR}asound.state /home/sferix/asound.state.default
cp ${DIR}asound.txt /home/sferix/asound.txt

# Install toga
echo 'Installing Toga'
tar -xvf ${DIR}toga.arm.bin.tar -C /home/sferix

# Setup toga folders on ramdisk
mkdir /media/ram/public_html
mkdir /media/ram/sferics
mkdir /home/sferix/public_html_static
ln -s /media/ram/sferics /home/sferix/sferics
ln -s /media/ram/public_html /home/sferix/public_html
touch /home/sferix/sferics/sferics.log
chown -R sferix /home/sferix
chown -R sferix /home/sferix/sferics
chown -R sferix /media/ram/sferics
chown -R sferix /media/ram/public_html
chown sferix /home/sferix/sferics/sferics.log

# Create folders at startup
cp ${DIR}ramdisk.sh /etc/init.d/
ln -s /etc/init.d/ramdisk.sh /etc/rc5.d/S90ramdisk

# Set /dev/snd ownership at startup
cp ${DIR}setsnd.sh /etc/init.d/
ln -s /etc/init.d/setsnd.sh /etc/rc5.d/S90setsnd

# Install TSIP programs
mkdir /home/sferix/gps
cp ${DIR}readTSIP.py /home/sferix/gps
cp ${DIR}sendTSIP.py /home/sferix/gps
cp ${DIR}startGPSD.py /home/sferix/gps
chown -R sferix /home/sferix/gps

# Install sferix crontab
crontab -u sferix ${DIR}crontab_install

# Create gps.log file
touch /home/sferix/public_html/gps.log
chown sferix /home/sferix/public_html/gps.log

# Add flash4 into authorized keys
mkdir /home/sferix/.ssh
cp ${DIR}authorized_keys /home/sferix/.ssh
chown sferix /home/sferix/.ssh
chown sferix /home/sferix/.ssh/authorized_keys

# Link public_html to htdocs
mv /usr/share/apache2/htdocs /usr/share/apache2/htdocs.orig
ln -s /home/sferix/public_html /usr/share/apache2/htdocs
chmod a+rx /home/sferix/public_html
cp ${DIR}index.html /home/sferix/public_html_static/

# Install preamp startup scripts
echo "Installing preamp scripts"
mkdir /home/sferix/preamp
cp ${DIR}preamp* /home/sferix/preamp/
chown sferix /home/sferix/preamp
chown sferix /home/sferix/preamp/*
cp ${DIR}preamp.sh /etc/init.d/
ln -s /etc/init.d/preamp.sh /etc/rc5.d/S90preamp

echo 'Reboot recommended'

\end{verbatim}

\section{WWLLN Service Unit v4 Initial Setup}

\subsection*{Method 1: SSH Setup}
The SSH setup method requires:
\begin{itemize}
\item{Ethernet cable}
\item{SSH capable computer}
\end{itemize}

\begin{enumerate}
\item{Connect SU to a host computer directly with an ethernet cable}
\item{Set host computer ethernet network settings to:
\begin{verbatim}
address:	192.168.10.1
gateway:	192.168.10.100
netmask:	255.255.255.0
\end{verbatim}}
\item{SSH into the SU from host computer:
\begin{verbatim}
ssh -p 7777 sferix@192.168.10.2
password: [	]
\end{verbatim}}
\item{Set desired static ip configuration in file $\sim$/networkSetup.sh}
\item{\begin{verbatim}
sudo ./networkSetup.sh
\end{verbatim}}
\item{Switch SU to main network ethernet within 1 minute of running networkSetup.sh}
\item{Test connection by SSH'ing into SU with new IP address}
\item{
\begin{enumerate}
\item{If successful: set new IP setting in /etc/network/interfaces}
\item{If unsuccessful: power cycle SU and check settings starting with step 3}
\end{enumerate}}
\item{Reset SU and confirm new settings}
\end{enumerate}

\subsection*{Method 2: Workstation Setup}
The Workstation setup method requires:
\begin{itemize}
\item{HDMI Monitor and cable}
\item{{\bf Powered} USB Hub}
\item{USB Keyboard}
\item{USB Mouse}
\end{itemize}

\begin{enumerate}
\item{Connect an HDMI display, keyboard and mouse}
\item{Set network information through GUI}
\end{enumerate}

\begin{centering}
\section*{Website Setup}
\end{centering}

\subsection*{Starting apache2}

To get apache2 running only one change needs to be made in the /etc/apache2/httpd.conf file.

\begin{verbatim}
Line 96:	#ServerName www.example.com:80
\end{verbatim}

Needs to be uncommented and changed to the hostname of the computer, e.g.:

\begin{verbatim}
Line96:	ServerName bobholz-3.ess.washington.edu:80
\end{verbatim}

Then httpd needs to be restarted:

\begin{verbatim}
sudo httpd -k restart
\end{verbatim}

\subsection*{Setting up the website}

All changes to the website need to be made in the /home/sferix/public\_html\_static folder, this folder is copied to /home/sferix/public\_html during start up.
Changes to public\_html are not saves as the folder is located in system RAM due to SD card read/write limitations.
A restart in not necessary if the public\_html\_static contents are copied to public\_html.


