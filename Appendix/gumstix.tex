\section{Hardware}

\subsection{Hardware Summary}

\subsection{Pinouts}

\subsection{Online Resources}

\section{Gumstix Operating System}

\subsection{\r{A}ngstrom}

The Gumstix used in the service unit is running the basic \r{A}ngstrom operating system (http://www.angstrom-distribution.org).
It also supports Ubuntu, however Ubuntu has been found to be slow and harder to configure with the Gumstix.
\r{A}ngstrom runs similar to most unix operating systems, with the main difference being the opkg package manager instead of yum or apt-get.

A useful resource in setting up and configuring the Gumstix software is the Gumstix developer site (http://gumstix.org) and the mailing list archive forum (http://gumstix.8.x6.nabble.com).

\subsection{Distribution Location}
\label{app:gum:distribution}
The operating system used is the Sakoman GNOME r13 build, available online:

http://www2.sakoman.com/category/8-gnome-daily-builds-r13.html

http://feeds.sakoman.com/feeds/gnome-r13/images/

Or in the /home/mlhutch/Git/gumstix.git repository on flash5 in the baseKernel folder.

\subsection{Bitbake and the Kernel}

The default kernel provided in the builds listed above is missing the ability to run ipfilter which is necessary as the firewall on the Gumstix operating system.
To enable this the kernel needs to be reconfigured to run netfilter, then compiled and deployed.
Bitbake was used previously to alter the linux kernel.

To set up a Bitbake development environment it is recommended to follow a guide such as:
http://ezrover.com/2012/01/12/sakoman-r13-gnome-firmware-howto-101-guide-for-gumstix-overo-openembedded-board/
Where the git repository for the bitbake and recipes is either:
www.sakoman.com/git/openembedded.git
or already pre-configured with the netfilter options on flash5 in the /home/mlhutch/Git/openembedded.git repository.

The following commands allow for changing the kernel:

\begin{enumerate}

\item \begin{verbatim}cd openembedded \end{verbatim}
\item \begin{verbatim}make ARCH=arm menuconfig \end{verbatim}

\item Enable all netfilter options under: Networking Support / Networking Options / Network Packet Filtering (netfilter) /

\item save results as new.config
\item Copy (and backup originals) as defconfig in: recipes/linux/linux-sakoman-pm-3.0/omap3-multi/defconfig

\item \begin{verbatim}bitbake virtual/kernel -c clean \end{verbatim}
\item \begin{verbatim}bitbake virtual/kernel \end{verbatim}

\end{enumerate}

Where the resulting kernel and modules are found in the tmp/deploy/glibc/images/overo/ directory.

After deploying a new kernel it may be necessary to recover the default boot variables on the Gumstix.
On the initial Gumstix startup, pause in the first 5 seconds (this needs to be done via the USB-console connection).
Once paused the defaults can be restored with:

\begin{verbatim}
nand erase 240000 20000
reset
\end{verbatim}

\section{Software}

\subsection{WWLLN Software}

The WWLLN software is provided by James Brundell and compiled specifically for the ARM process.
The three programs are {\bf toga}, {\bf ntpcheck}, and {\bf GDspectro}.
{\bf toga} is the main WWLLN processing programming that reads in the VLF and GPS signals to produce the UDP packets sent on to the main WWLLN processors.
It should be always running on the system with a crontab entry such as:

\begin{verbatim}
0,5,10,15,20,25,30,35,40,45,50,55 * * * * toga -s 100 -a 3 -j 1 -g -o &
\end{verbatim}

This will try to start it every 5 minutes in case it stops for any reason.

The {\bf ntpcheck} and {\bf GDspectro} and programs used by {\bf toga} but do not need to be called or run on their own.

\subsection{Hardware Controls}

Pin GPIO145 is the pin that controls whether the preamp power supply is turned on or off.
When the pin is held low (value of 0) the power supply is on, when it is set high (value of 1) it is turned off.
The command to change a GPIO pin value is:

\begin{verbatim}
echo 0 > /sys/class/gpio/gpio145/value
\end{verbatim}

The two scripts {\bf preampOn.sh} and {\bf preampOff.sh} can be used to easily toggle the preamp power supply.

The default value for GPIO pins is to hold them high, so during boot the preamp turns off until the {\bf preampOn.sh} script can be called at the end of the boot sequence.

\subsection{GPS Interface}

The Trimble GPS communicates with the TSIP protocal, compared the NMEA of the previous GPS engine used.
The pythons script {\bf readTSIP.py} interprets the TSIP messages and reports the GPS status to the file gps.log and prints them to the console.
The console printing can be turned off by changing the variable print\_to\_console to False.

The program can be started and run in the background to produce a continues record of GPS activity.
The default location for the gps.log file is in the public\_html folder where it can be remotely checked through the service unit website.

\subsection{RAM Disk}

The \r{A}ngstrom distribution for Gumstix automatically sets up a RAM disk for users.
It is created at /media/ram with half of the available RAM (256~MB).
It needs to be used for the running of the WWLLN software as the microSD card is too slow.
At start-up the public\_html folder (logs and spectrograms) and sferics folder are created in the ram disk and symlinked to the main sferics directory.

For this reason all permanent edits to the Service Unit website should be made in the public\_html\_static directory.

\section{Creating Gumstix microSD Card}

There are two methods for configuring a new microSD card for use with the service unit Gumstix computer.
Either a card can be formatted and loaded with the latest software using the {\bf makeSD.sh} script, or an existing installation disk image can be copied over.

The {\bf makeSD.sh} is based on the script given in Subsection~\ref{app:gum:distribution}.
It is configured to run on a Linux system but may also work with other Unix based operating systems.
The script automatically formats the microSD, loads the bootscripts, loads the operating system, and updates the kernel and modules produced with Bitbake.

Once a new card is created it is advised to make an exact copy of the card to allow for duplication onto new cards.
Since the partitioning, and bit location on the card, is critical for the Gumstix the direct copy program {\bf dd} should be used to make the copy.
To copy a card to a disk image the command will look like:

\begin{verbatim}
sudo dd if=/dev/sdb of=/Path/To/Target/gumstix.iso bs=512
\end{verbatim}

And the command to copy a microSD image back to a new card in the same slot as the previous card:

\begin{verbatim}
sudo dd if=/Path/To/Target/gumstix.iso of=/dev/sdb bs=512
\end{verbatim}

A word of warning: the {\bf dd} command can overwrite and destroy hard drives if they are incorrectly targeted.
Always double check the mount point of the microSD card before running the {\bf dd} command.

\section{Gumstix System Setup}

\subsection{Connecting to the Gumstix}

\subsection{Installing Packages}

%\lstinputlisting{../Gumstix/firstRunFiles/install.sh}

\subsection{Configuring Settings}

%\lstinputlisting{../Gumstix/firstRunFiles/setup.sh}

\subsection{Sound Settings}
