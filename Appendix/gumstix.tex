\section{Hardware}

The WWLLN Service Unit uses a Gumstix WaterStormCOM mounted on a Tobi breakout board as the on-board computer to run the WWLLN software.

\subsection{Hardware Summary}

The Gumstix WaterStormCOM is part of the Overo COM series and has the following features:

\begin{itemize}
\item 1~GHz ARM Cortex-A8 CPU
\item 512~MB RAM
\item microSD Card Slot
\item OpenGL POWER SGX graphics Aaccelerator
\item C64x Fixed Point DSP (Max: 660,800~MHz)
\end{itemize}

The Tobi breakout boards adds:

\begin{itemize}
\item HDMI video Out
\item 1 USB port
\item 1 USB console connection
\item Stereo in/out
\item Ethernet
\end{itemize}

\subsection{Pinouts}

The Gumstix COM has direct pinouts to the processor, however as it is not used without the Tobi breakout board, only the Tobi pinout is shown in Figure~\ref{app:gumstix:fig:tobi}.

\begin{figure}[ht!]
   \centering
   \includegraphics[scale=1]{Appendix/Figures/tobi_pinout.pdf}
   \caption{Pinout of the Gumstix Tobi breakout board.}
   \label{app:gumstix:fig:tobi}
\end{figure}

In the service unit only pins 1 (GND), 9 (RXD1), 10 (TXD1), 28 (GPIO145), and 40 (V\_BATT) are used.
RXD1 and TXD1 are used for the serial communication with the GPS.
GPIO145 is used to control the pre-amp power supply.
V\_BATT is the 5~V power for the Gumstix.

\section{Gumstix Operating System v2.0}

\subsection{Gumstix-Yocto}

The Gumstix used in the service unit is running a custom Linux distribution created using the Yocto Project build system.
This OS runs similar to most unix operating systems, with the main difference being the smart package manager instead of yum or apt-get.

A useful resource in setting up and configuring the Gumstix software is the Gumstix developer site (http://gumstix.org) and the mailing list archive forum (http://gumstix.8.x6.nabble.com).

\subsection{Distribution Location}
\label{app:gum:distribution}
The operating system used is the Yocto Project Gumstix Layer v1.5, available on flash5 or in the Git repository:

/home/mlhutch/Git/gumstix.git repository on flash5 in the image folder.

\subsection{Building}

Building the operating system can be done by following the instructions at:

https://github.com/gumstix/Gumstix-YoctoProject-Repo

With the final configuration in the Git repository on flash5 in image/yocto/local.conf.
If rebuilding use the local.conf file in the build/conf folder and run the bitbake target \texttt{bitbake gumstix-xfce-image}.

With either a newly made OS image or the one in the gumstix.git repository, follow the install instructions in INSTALL.md for setting up a new microSD card to run with WWLLN.

\section{Software}

\subsection{WWLLN Software}

The WWLLN software is provided by James Brundell and compiled specifically for the ARM process.
The three programs are {\bf toga}, {\bf ntpcheck}, and {\bf GDspectro}.
{\bf toga} is the main WWLLN processing programming that reads in the VLF and GPS signals to produce the UDP packets sent on to the main WWLLN processors.
It should be always running on the system with a crontab entry such as:

\begin{verbatim}
0,5,10,15,20,25,30,35,40,45,50,55 * * * * toga -s 100 -a 3 -j 1 -g -o &
\end{verbatim}

This will try to start it every 5 minutes in case it stops for any reason.

The {\bf ntpcheck} and {\bf GDspectro} and programs used by {\bf toga} but do not need to be called or run on their own.

\subsection{Hardware Controls}

Pin GPIO145 is the pin that controls whether the preamp power supply is turned on or off.
When the pin is held low (value of 0) the power supply is on, when it is set high (value of 1) it is turned off.
The command to change a GPIO pin value is:

\begin{verbatim}
echo 0 > /sys/class/gpio/gpio145/value
\end{verbatim}

The two scripts {\bf preampOn.sh} and {\bf preampOff.sh} can be used to easily toggle the preamp power supply.

The default value for GPIO pins is to hold them high, so during boot the preamp turns off until the {\bf preampOn.sh} script can be called at the end of the boot sequence.

\subsection{GPS Interface}

The Trimble GPS communicates with the TSIP protocal, compared the NMEA of the previous GPS engine used.
The pythons script {\bf readTSIP.py} interprets the TSIP messages and reports the GPS status to the file gps.log and prints them to the console.
The console printing can be turned off by changing the variable print\_to\_console to False.

The program can be started and run in the background to produce a continues record of GPS activity.
The default location for the gps.log file is in the public\_html folder where it can be remotely checked through the service unit website.

\subsection{RAM Disk}

The Linux distribution for Gumstix automatically sets up a RAM disk for users.
It is created at /var/volatile with half of the available RAM (256~MB).
It needs to be used for the running of the WWLLN software as the microSD card is too slow.
At start-up the public\_html folder (logs and spectrograms) and sferics folder are created in the ram disk and symlinked to the main sferics directory.

For this reason all permanent edits to the Service Unit website should be made in the public\_html\_static directory.

\section{Creating Gumstix microSD Card}

There are two methods for configuring a new microSD card for use with the service unit Gumstix computer.
Either a card can be formatted and loaded with the latest software following the INSTALL.md instructions, or an existing installation disk image can be copied over (located on flashfile and flash5).

\subsection{Card Duplication}
\label{app:gumstix:duplication}

Once a new card is created it is advised to make an exact copy of the card to allow for duplication onto new cards.
Since the partitioning, and bit location on the card, is critical for the Gumstix the direct copy program {\bf dd} should be used to make the copy.
To copy a card to a disk image the command will look like:

\begin{verbatim}
sudo dd if=/dev/sdb of=/Path/To/Target/gumstix.iso bs=512
\end{verbatim}

And the command to copy a microSD image back to a new card in the same slot as the previous card:

\begin{verbatim}
sudo dd if=/Path/To/Target/gumstix.iso of=/dev/sdb bs=512
\end{verbatim}

A word of warning: the {\bf dd} command can overwrite and destroy hard drives if they are incorrectly targeted.
Always double check the mount point of the microSD card before running the {\bf dd} command.

\section{Gumstix System Setup}

Subsection~\ref{app:appendix:setup} gives the setup instructions for a new station, namely setting the IP address so it can be added to the network.
A more up to date version is available in the {\bf gumstix.git} repository in the {\bf user\_manual} folder.
Also included in the station setup instructions are how to setup and customize the service unit webpage, as described in Subsection~\ref{app:gumstix:website}.

\subsection{Connecting to the Gumstix}
\label{app:appendix:setup}
%%
%% Update to latest version

\subsection*{Method 1: SSH Setup}
The SSH setup method requires:
\begin{itemize}
\item{Ethernet cable}
\item{SSH capable computer}
\end{itemize}

\begin{enumerate}
\item{Connect SU to a host computer directly with an ethernet cable}
\item{Set host computer ethernet network settings to:
\begin{verbatim}
address:	192.168.10.1
gateway:	192.168.10.100
netmask:	255.255.255.0
\end{verbatim}}
\item{SSH into the SU from host computer:
\begin{verbatim}
ssh -p 7777 sferix@192.168.10.2
password: [	]
\end{verbatim}}
\item{Set desired static ip configuration in file $\sim$/networkSetup.sh}
\item{\begin{verbatim}
sudo ./networkSetup.sh
\end{verbatim}}
\item{Switch SU to main network ethernet within 1 minute of running networkSetup.sh}
\item{Test connection by SSH'ing into SU with new IP address}
\item{
\begin{enumerate}
\item{If successful: set new IP setting in /etc/network/interfaces}
\item{If unsuccessful: power cycle SU and check settings starting with step 3}
\end{enumerate}}
\item{Reset SU and confirm new settings}
\end{enumerate}

\subsection*{Method 2: Workstation Setup}
The Workstation setup method requires:
\begin{itemize}
\item{HDMI Monitor and cable}
\item{{\bf Powered} USB Hub}
\item{USB Keyboard}
\item{USB Mouse}
\end{itemize}

Connect the powered USB hub to the back USB port of the service unit, and attach the keyboard and mouse to the hub.
Connect a monitor to the HDMI port, DVI - HDMI adapters work as well.
Power on the box, it will take a few minutes for the login screen to show up.
Select ``Other...'' and login with the username \textbf{host}.
Wait a few more minutes for the graphical display to load.

Adjust the network settings by the steps listed in Method 1 or 3.

\begin{enumerate}
\item{Connect an HDMI display, keyboard and mouse}
\item{Follow Method~1 or Method~3}
\end{enumerate}


\subsection*{Method 3: Manual microSD Editing}

The files that need to be edited on the rootfs partition are:

\begin{verbatim}
/etc/network/interfaces
/etc/resolv.conf
/etc/systemd/system/sshd.socket
\end{verbatim}

The interfaces file lists the IP information of the machine whole the resolv.conf file is for the DNS information.
The sshd.socket file sets the port with which SSH is allowed.

The last step, if a non-standard port is being used, is to also alter the built in firewall of iptables and netfilter.
The firewall settings are stored in:

\begin{verbatim}
/etc/iptable.rules
\end{verbatim}

and can be edited as a standard iptables configuration file.

\subsection{Website Setup}
\label{app:gumstix:website}

\subsection*{Starting apache2}

To get apache2 running only one change needs to be made in the /etc/apache2/httpd.conf file.

\begin{verbatim}
#ServerName www.example.com:80
\end{verbatim}

Needs to be uncommented and changed to the hostname of the computer, e.g.:

\begin{verbatim}
ServerName gumstix.ess.washington.edu:80
\end{verbatim}

Then httpd needs to be restarted:

\begin{verbatim}
sudo httpd -k restart
\end{verbatim}

\subsection*{Setting up the website}

All changes to the website need to be made in the /home/sferix/public\_html\_static folder, this folder is copied to /home/sferix/public\_html during start up. Changes to public\_html are not saves as the folder is located in system RAM due to SD card read/write limitations. A restart in not necessary if the public\_html\_static contents are copied to public\_html.

\subsection{Sound Settings}

The Gumstix a myriad of analog inputs that are all controlled with {\bf alsamixer}.
For the WWLLN service unit the stereo input is controlled by the $<$Analog$>$ input as shown highlighted in Figure~\ref{app:gumstix:fig:alsa}, and a digital gain through $<$TX1 Digital$>$.
Included in the installation files is the default alsa profile {\bf asound.state}.

\begin{figure}[ht!]
   \centering
   \includegraphics[scale=.25]{Appendix/Figures/gumstixmixer.png}
   \caption{Alsamixer settings for Gumstix stereo input, $<$Analog$>$ controls the stereo gain.}
   \label{app:gumstix:fig:alsa}
\end{figure}

\section{Common Problems}

\begin{itemize}
\item The network icon in the top menu bar says the network connections are disabled.

\emph{The GUI network manager is disabled, but the network settings set as above still work.}

\item The GPS pulse per second is not working with the TOGA program, it lists ``PPS bad'' for most lines.

\emph{Adjust the gain on the pulse per second with Alsamixer <TX1 Digital> right channel, usually lowering it will resolve the problem.}

\end{itemize}


