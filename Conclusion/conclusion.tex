Placeholder for conclusion

\section{WWLLN}

\subsection{Network Characterization}

Network characterization:
  Leave one out detection efficiency study for station placement
  Variability in regional energy distributions over time
  Energy uncertainty as distance from main calibrated station
  Speed of light optimization
  Regional comparisons with CHUVA, ENTLN, GOES-R
  Flash multiplicity and subsequent stroke energies
\subsection{Network Improvements}

Network improvements
  Additional calibrated stations for energy bootstrap calculation
  Optimum station placement prediction
  Relative detection efficiency using more robust measure

\section{Thunderstorms}

\subsection{Cluster Validation}

Cluster comparison to weather radar and case studies
  Thunderstorm property Variability

\subsection{Parameterization}

  Thunderstorm parameterization and controls of flash rate and stroke energy
  Analysis of thunderstorm charging rate

\subsection{Thunderstorm Properties}

 Thunderstorm properties
  Area, durations over time over region with maps
  Measure of electrical activity of thunderstorms
     e.g. total flash energy per unit area per unit time
     Check land-ocean difference
  Thunderstorm total stroke count average maps
  Thuderstorm properties of different thunderstorm types
  Case studies with radar data for parameterization

\section{Global Electric circuit}

\subsection{Model Improvement}

\subsection{Validation Campaign}

Global electric circuit balloon campaign and comparison to WWLLN
  Stroke count model, flash count model, thunderstorm count model

\section{Lightning Processes}

\subsection{Whistlers}

  Automatic whistler detection

Whistler generation and coupling into the ionosphere, measurements with satellites such as RBSP

\subsection{Terrestrial Gamma-Ray Flashes}

 TGF production estimates using energy of detected TGFs





