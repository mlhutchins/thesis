Over the course of this work WWLLN has been expanded to include several new features: energy per stroke, relative detection efficiency, flash clustering, and thunderstorm clustering.
These new features have been developed alongside new research and analysis using WWLLN by itself or in conjunction with other systems.
Most of the research completed naturally leads to more projects and research avenues that can be explored in the context of this work.
Several of the possibilites for research on WWLLN, thunderstorms, the global electric circuit, and other lightning processes are briefly discussed in this chapter.

\section{WWLLN}

The relative detection efficiency model developed in Chapter~\ref{thesis:chapter:efficiency} provides one view of the network performance but it relies on several assumptions.
Either the model can be advanced to not rely on these assumptions or those assumptions themselves can be monitored.
The main assumption is that the distribution of lightning energy is the same everywhere, this was shown to not be the case in Chapter~\ref{thesis:chapter:landsea} when examining land and oceanic thunderstorms.
The model can take into account the varying distributions, or the expected variance in the energy distribution, by looking into the variability of the observerd regional energy distributions over time.

A promising method for examining the energy detection performance is to check how the measured energy values change when the network changes station configuration.
For example, processing the energy data a second time after artificially removing a few stations to quantify how the energy distributions change.
Compare the energy distribution changes to the change relative to another network (e.g. ENTLN or LIS) to better paramterize the detection efficiency model.

Similarly the accuracy of the energy measurements themselves can be improved by adapting the energy per stroke processing to include multiple well calibrated WWLLN stations.
In the processing described in this work the entire network is bootstrap calibrated from one station; the addition of a second well calibrated station can improve the energy measurements and help quantify the regional variation in energy uncertainty.
This is assuming that more uncertainty is introduced in a stations calibrated as the station requires more calibration steps from the first station.

Finally other network characterizations can be developed based on the clustering results.
The number of thunderstorms, the average number of strokes per thunderstorm, average flash multiplicity, or time between flashes within a single thunderstorm.

\section{Thunderstorms}

\subsection{Cluster Validation}

The thunderstorm clustering discussed in Chapter~\ref{thesis:chapter:gec} and~\ref{thesis:chapter:thunderstorm} can benefit from additional validation.
Two potential methods for validation could be a large scale comparison to TRMM precipitation and case studies against regional weather radar.
The TRMM study would require grouping of the precitipation data and a comparison of those groupings to the thunderstorm areas.
Weather radar comparisons would provide detailed ground truth comparisons of the thunderstorm extent and duration to the measured radar data.
Further weather radar would allow for a measure of thunderstorm tracking accuracy.

Aside from comparisons to other sytems the properties measured by the thunderstorm clustering can be tracked over spatial and temporal scales.
For example the average thunderstorm duration interannual variability in a given region or the overall network observed thunderstorm duration.
This can explore the effect of the growing network on the measured values and the accuracy of the clustering algorithm as the stroke rate detected by WWLLN changes with time.

\subsection{Parameterization}

Thunderstorm parameterizations are functional relationships between different properties of thunderstorms.
Using lightning detection networks allows for the development of these: inferring thunderstorm properties based on the observed lightning flash rates, in particular for regions with low availibility of direct observations (e.g. radar, satellite).
\citet{Zipser1994} examined the relation between the flash rate in a thunderstorm with the updraft velocities, which in turn are controlled by the differential surface heating below the thudnerstorm.
Parameterization of flash rate based on other observations has led to simple models of global lightning behavior, \citet{Price1992} used cloud height parameterization for flash rate.
These empirical parameterizations can help in the development and validation of models (e.g. \citet{Baker1999}) of thunderstorm electrification and enable or expand prediction of the global electric circuit, NO$_\text{x}$ production, and weather forecasting.
Investigating the parameterization of lightning behaviour in relation to the properties of a thunderstorm on a large scale, compared to small studies, can create better and more robust paramterizations.
The WWLLN thunderstorm clustering results can be compared to other networks and systems in order to develop and validate parameterizations of inferred thunderstorm parameters.

With the development of these parameterizations the WWLLN clustering can be used to constrain the various controls on the flash rate and subsequent flash energy.
From there the thunderstorm charging rate can be estimated, allowing for research on the difference between continental and oceanic thunderstorms.

\subsection{Thunderstorm Properties}

The WWLLN thunderstorm clustering can be used to investigate global and regional thunderstorm properties and their distributions.
Thunderstorm area and duration are briefly discussed in Chapter~\ref{thesis:chapter:thunderstorm}, but can be explored in more depth and in more regions.
These two properties can be measured over different spatial and temporal scales.
For example the distribution of winter thunderstorm over the Sea of Japan compared to the Gulf of Mexico, or the South American ``green ocean'' phenoma in thunderstorm behavior.

Alternativey more complex measures of thundertorms can be developed.
The average stroke count per thunderstorm, the peak flash rate, parameterizations, or the electrical activity.
The electrical activity can be measured as the total flash energy per unit area per unit time for the thunderstorm: the sum of measured flash energies within the thunderstorm.
This measure, and others, may be highly dependent on the network performance or may not be absolute calibrated.
However they can still allow for relative comparisons between thunderstorms in a limited temporal or spatial scope (e.g. North America in July 2013).

The thunderstorm dataset can also be advanced by classifying thunderstorms into different thunderstorm types.
While this may only work for small case studies it may be possible to use more advanced machine learning classifiers to automatically assign cloud types to clusters.

\section{Global Electric circuit}

The global electric circuit model presented in Chapter~\ref{thesis:chapter:gec} provides an estimate of the thunderstorm contribution to the circuit based on the WWLLN thunderstorm clusters.
It is shown that the model is able to reproduce the expected Carnegie curve over lone time scales and has the capability to provide current estimates on shorter time scales.
The model can be improved for a more accurate measure of the current contribution, but without any validation it cannot be reliable used.

For the model improvements there are several underlying assumptions that can be factored in.
The main one is that all thunderstorms produce the same upward current to the ionosphere regardless of their size or activity.
The estimates used in the model from \citep{Mach2010} can be scaled by either the thunderstorm cluster activity or the estimated cluster size.
Since the development of the model the thunderstorm cluster area was validated against ENTLN in Chapter~\ref{thesis:chapter:thunderstorm}, giving confidence to the area measurements.
The second assumption is that all continental thunderstorms will behave the same as well all oceanic thunderstorms.
This assumption is harder to address as it relies on direct measurements made in different regions.

Regardless of the model improvements made, the model needs to be compared against a ground truth measurement.
Further, comparison to a ground truth will allow for adjustments of the model to better fit the observerd measurements, or if adjustments cannot be made then other components of the global electric circuit need to be included in the model.
For the ground truth a balloon campaign needs to be conducted to measure the fair weather return current over the span of one to several weeks.
With a reliable and clean measurements of the return current the model validation can be performaed on a very short time scale instead of relying on long term averages.

\section{Other Lightning Processes}

WWLLN is able to measure and detect more than just the main discharge of a lightning stroke.
The network is able to directly detect terrestrial gamma-ray flashes (TGF) and their related strokes, it is also able to detect whistler waves at the individual stations.
For TGFs WWLLN has been used to geolocate either the TGF itself or the thunderstorm it originated from.
WWLLN can be used to create an estimate of the global TGF production rate by comparing the detected TGFs with the lightning observed by one of the TGF satellites (e.g. FERMI or RHESSI).
With whistler waves an automated whistler detector running at the remote WWLLN stations could enable the creation of a large database of whistler waves.
Such a database could provide the means for investigations of the ionosphere and magnetosphere, especially if the detector is able to run on multiple WWLLN stations.







