Over the course of this work WWLLN has been expanded to include several new features: energy per stroke, relative detection efficiency, flash clustering, and thunderstorm clustering.
These new features have been developed alongside new research and analysis using WWLLN by itself or in conjunction with other systems.
Most of the research completed naturally leads to more projects and research avenues that can be explored in the context of this work.
Several of the possibilites for research on WWLLN, thunderstorms, the global electric circuit, and other lightning processes are briefly discussed in this chapter.

\section{WWLLN}

The relative detection efficiency model developed in Chapter~\ref{thesis:chapter:efficiency} provides one view of the network performance but it relies on several assumptions.
Either the model can be advanced to not rely on these assumptions or those assumptions themselves can be monitored.
The main assumption is that the distribution of lightning energy is the same everywhere, this was shown to not be the case in Chapter~\ref{thesis:chapter:landsea} when examining land and oceanic thunderstorms.
The model can take into account the varying distributions, or the expected variance in the energy distribution, by looking into the variability of the observerd regional energy distributions over time.

A promising method for examining the energy detection performance is to check how the measured energy values change when the network changes station configuration.
For example, processing the energy data a second time after artificially removing a few stations to quantify how the energy distributions change.
Compare the energy distribution changes to the change relative to another network (e.g. ENTLN or LIS) to better paramterize the detection efficiency model.

Similarly the accuracy of the energy measurements themselves can be improved by adapting the energy per stroke processing to include multiple well calibrated WWLLN stations.
In the processing described in this work the entire network is bootstrap calibrated from one station; the addition of a second well calibrated station can improve the energy measurements and help quantify the regional variation in energy uncertainty.
This is assuming that more uncertainty is introduced in a stations calibrated as the station requires more calibration steps from the first station.

Finally other network characterizations can be developed based on the clustering results.
The number of thunderstorms, the average number of strokes per thunderstorm, average flash multiplicity, or time between flashes within a single thunderstorm.

\section{Thunderstorms}

\subsection{Cluster Validation}

Cluster comparison to weather radar and case studies
  Thunderstorm property Variability

\subsection{Parameterization}

  Thunderstorm parameterization and controls of flash rate and stroke energy
  Analysis of thunderstorm charging rate

\subsection{Thunderstorm Properties}

 Thunderstorm properties
  Area, durations over time over region with maps
  Measure of electrical activity of thunderstorms
     e.g. total flash energy per unit area per unit time
     Check land-ocean difference
  Thunderstorm total stroke count average maps
  Thuderstorm properties of different thunderstorm types
  Case studies with radar data for parameterization

\section{Global Electric circuit}

\subsection{Model Improvement}

\subsection{Validation Campaign}

Global electric circuit balloon campaign and comparison to WWLLN
  Stroke count model, flash count model, thunderstorm count model

\section{Lightning Processes}

\subsection{Whistlers}

  Automatic whistler detection

Whistler generation and coupling into the ionosphere, measurements with satellites such as RBSP

\subsection{Terrestrial Gamma-Ray Flashes}

 TGF production estimates using energy of detected TGFs





